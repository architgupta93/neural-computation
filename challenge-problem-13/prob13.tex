\documentclass[10pt,onecolumn]{article}
% Package includes handled in a separate file
\title{VS265 Problem 13: Why spikes?}
\author{Archit Gupta}
\date{}
\input package-includes
\input acronyms
\begin{document}
    \maketitle
    \vspace{-2em}
    \noindent\rule{\textwidth}{1.4pt}
    \vspace{1em}

    \section*{Problem description}
    \label{sec:problem_description}

    The \ac{CNS} in vast majority of animals employs action potentials or spikes as the primary mode of communication.
    In this challenge problem, we discuss some of the biological constraints that might have made spiking necessary.
    We also compare the \ac{CNS} of simpler organisms like \textit{C. Elegans} which do not employ spikes.

    \section*{Essay}
    \label{sec:essay}

    Before we take a mechanistic view towards computation and signaling in the brain, we need to discuss the role that the brain, or \ac{CNS} at large plays in an organism's existence.
    In~\cite{sterling2015principles}, the authors discuss the role of a computational organ, or more generally computational units in different organisms.
    A brain serves to aggregate sensory information that is relevant to an animal's behavior and then, having processed the sensory information, it facilitates actions in the animal's behavioral repertoire.
    We also learn from~\cite{sterling2015principles} that larger animals are able to move faster which in turn allows them to explore and exploit their environment to a greater extent.
    As the complexity of actions and computational processing increases, information undergoes multiple levels of processing (for example in the human visual stream~\cite{felleman1991distributed}).
    Under these circumstances, the reliability, speed and bandwidth of information transfer become vital.
    We next look at some of these ideas
    \begin{enumerate}
        \item How can we reliably transfer information - the case for continuous signals \textit{vs.} discrete packets of information.
        \item Signaling over long distances - For larger organisms, information should propagate from the extremities to the brain and vice-versa at a behaviorally relevant timescale.
        \item The information transfer rate, or bandwidth, should be high enough to cope with the sensory modalities and behavioral demands of the animal,
        \item Energy spent in performing these operations, and the trade-off between energy expenditure and existential benefits?
    \end{enumerate}

    \subsection*{Continuous \textit{vs.} Discrete signaling}
    \label{sec:continuous_vs_discrete_signaling}
    The debate on continuous \textit{vs.} discrete signals as a substrate for computation has raged in the computational world for much of the last century.
    A single memory or computational unit carries more information as a continuous signal, than a discrete one.
    However, in the presence of noise, using more computational or memory units and a discrete (say binary) representation is more resilient to noise.
    The biological substrate for computation and signaling consists of ion channels and chemical pathways which are inherently stochastic in nature.
    Because of the small volume of molecules involved in individual computations, the \ac{SNR} is often low for individual computational units.
    This makes a discrete signal representation a better choice.

    \subsection*{Propagating signals over long distances}
    \label{sec:propagating_signals_over_long_distances}

    Electrical models of transmission lines suggest that when propagated over long distances, continuous-time signals suffer from dispersion and attenuation~\cite{roychowdhury1991efficient}.
    In the presence of noise, the signal quality is further degraded.
    Several organisms, mammals in particular, have grown to immense proportions.
    The spinal cord of a Giraffe can be upto $2.6$m long~\cite{badlangana2007observations}.
    Signals from the brain to motor neurons must travel long distances.
    In that regard, discrete packets like action potentials are less susceptible to attenuation and dispersion and therefore, more reliable for information exchange.

    \subsection*{Capturing rapidly changing information}
    \label{sec:capturing_rapidly_changing_information}

    Our discussion so far suggests that discrete signals can more reliably communicate information from sensory areas and commands to motor areas.
    Organisms often need to adapt to fast changing environmental variables like sight and sound.
    In order to cope with fast changing environmental input, neurons must be able to operate at higher frequencies.
    This necessitates a short packet of information which can then be repeated at a fast time scale in order to convey more than all-or-none information.
    Trains of action potentials, or spike trains efficiently accomplish this.
    In~\cite{mainen1995reliability}, the authors show that spike timing (or inter-spike intervals) can effectively capture fluctuations in the inputs.

    \subsection*{The case for simpler organisms}
    \label{sec:the_case_for_simpler_organisms}
    Let us take the case of a simpler organism.
    C. Elegans can grow upto $1.5$mm long~\cite{riddle1997introduction}.
    Further, they lack a centralized organ like the brain where all sensory information is collected and processed.
    Instead, collection of sensory information and computations that lead to motor or chemical actions are often performed locally.
    The sensory information that C. Elegans primarily encounter, like chemical concentrations and temperature gradients usually do not change rapidly with time.
    Because of that, slower computational processes are sufficient for the organism to be able to react to sensory information in time.
    Since the computational demands and the behavioral repertoire of this simpler organism are much limited, it does not need to expend energy in producing high-frequency actions potentials.

    \section*{Summary}
    \label{sec:summary}
    We discussed four major factors that need to be assessed in neural design - reliability, speed, bandwidth and energy expenditure.
    As the complexity of an animal and its behavioral demands increase, it can spend more energy and utilize spikes for more reliable and faster signaling.
    On the other hand, when the animal complexity and behavioral demands are low, simpler signaling mechanisms can be used.

    \setlength{\bibsep}{2pt}
    \bibliographystyle{unsrt}
    \footnotesize
    \bibliography{citations.bib}
\end{document}
